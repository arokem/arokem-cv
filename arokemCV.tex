
%\documentclass[11pt,article,oneside]{memoir}
\documentclass[11pt,fullpage]{article}

\usepackage{cite}
\usepackage{bibentry}
\usepackage[usenames,dvipsnames]{xcolor} % Custom colors
\usepackage[colorlinks]{hyperref}
\usepackage{geometry}
\usepackage{listliketab}
\usepackage{array}
\usepackage{longtable}
\usepackage{natbib}
\usepackage{fontspec}
\usepackage{marginnote} % For margin years
\usepackage{etaremune}
\usepackage{hyperref}
%\usepackage{libertine}
%\usepackage{ETbb}
\usepackage{mathpazo}
\newcommand{\years}[1]{\marginnote{\scriptsize #1}} % New command for including margin years
\renewcommand*{\raggedleftmarginnote}{}
\setlength{\marginparsep}{-1pt} % Slightly increase the distance of the margin years from the contant
\reversemarginpar


\usepackage{geometry}
\geometry{verbose,letterpaper,tmargin=2cm,bmargin=2cm,lmargin=2cm,rmargin=2cm}

% \setmainfont{Latin Modern Sans}
\setmainfont{Palatino}

\nobibliography*
\bibliographystyle{apalike}

\newcommand\tab[1][1cm]{\hspace*{#1}}

% Link colors
\hypersetup{linkcolor=SteelBlue,citecolor=blue,filecolor=black,urlcolor=SteelBlue}

% Palatino font
%\usepackage[T1]{fontenc}
%\usepackage[sc,osf]{mathpazo}

\def\name{Ariel Rokem}

\reversemarginpar

\geometry{
  body={6.5in, 8.5in},
  left=1.0in,
  top=1.0in
}

% Customize page headers
%\pagestyle{myheadings}
\usepackage{fancyhdr}
\pagestyle{fancy}
\rhead{\name}
\thispagestyle{empty}

% Custom section fonts
\usepackage{sectsty}
\sectionfont{\rmfamily\mdseries\Large}
\subsectionfont{\rmfamily\mdseries\itshape\large}

% Other possible font commands include:
% \ttfamily for teletype,
% \sffamily for sans serif,
% \bfseries for bold,
% \scshape for small caps,
% \normalsize, \large, \Large, \LARGE sizes.

% Don't indent paragraphs.
\setlength\parindent{0em}

% Make lists without bullets
\renewenvironment{itemize}{
  \begin{list}{}{
    \setlength{\leftmargin}{1.5em}
  }
}{
  \end{list}
}

\hypersetup{linkcolor=blue,citecolor=blue,filecolor=black,urlcolor=MidnightBlue}

\begin{document}
%\bibliographystyle{apalike}

% Print name centered and bold:
\centerline{\Large \bf \name}

\vspace{0.25in}

\begin{minipage}{0.50\linewidth}
  Department of Psychology\\
  The University of Washington\\
  119A Guthrie Hall\\
  Seattle, WA 98105\\
\end{minipage}
\begin{minipage}{0.50\linewidth}
  \begin{tabular}{ll}
    Phone: & +1-510-3876264 \\
    Email: & \href{mailto:arokem@gmail.com}{arokem@gmail.com} \\
    Homepage: & \href{http://arokem.org/}{www.arokem.org} \\
    ORCID: & \href{http://orcid.org/0000-0003-0679-1985}{0000-0003-0679-1985} \\
  \end{tabular}
\end{minipage}

\section*{Work Experience}
\begin{tabular}{ll}

  2021 -  present & {\bf Research Associate Professor} \\ & \emph{University of Washington Department of Psychology}\\
%  2021 -  present & {\bf Adjunct Associate Professor} \\ & \emph{Paul Allen School of Computer Science \& Engineering}\\
  2020 - 2021 & {\bf Research Assistant Professor} \\ & \emph{University of Washington Department of Psychology}\\
  2015-2020 & {\bf Senior Data Scientist}\\ & \emph{University of Washington eScience Institute}\\
  2011-2015 & {\bf Postdoctoral Researcher}\\ & \emph{Stanford University}\\
  2010-2011 & {\bf Postdoctoral Researcher}\\ & \emph{University of California, Berkeley}\\
  2002-2003 & {\bf Research Student}\\ & \emph{Humboldt-Universit\"{a}t zu Berlin}\\

\end{tabular}

\section*{Education}

	2010  {\bf Ph.D. } Neuroscience, University of California, Berkeley \\
	2005  {\bf M.A.} (\emph{Summa cum Laude}) Cognitive Psychology, Hebrew University of Jerusalem \\
	2002  {\bf B.Sc.} (\emph{Cum Laude}) Biology and Psychology, Hebrew University of Jerusalem \\


\section*{Peer-reviewed publications}

\textbf{Google Scholar, total citations: 7,259, h-index: 40}

\begin{etaremune}
  \item \bibentry{Poldrack2024bids}
  \item \bibentry{Roy2024Reading}
  \item \bibentry{Cieslak2023HeadMotion}
  \item \bibentry{Caffara2023AlphaFA}
  \item \bibentry{Ferre2023NSDMD}
  \item \bibentry{Liu2023GPU}
  \item \bibentry{chang2023cdmri}
  \item \bibentry{kruper2023cdmri}
  \item \bibentry{Grotheer2023dhcp}
  \item \bibentry{Kruper2023OR}
  \item \bibentry{RichieHalfordCieslak2022HBNPOD2}
  \item \bibentry{Yucel2022Factors}
  \item \bibentry{Graham2022forestexplorR}
  \item \bibentry{fadnavis2021bifurcated}
  \item \bibentry{Hayot-Sasson2021-fn}
  \item \bibentry{rokem2021detect}
  \item \bibentry{Graham2021-ql}
  \item \bibentry{Levitis2021-yx}
  \item \bibentry{Caffarra2021WMEEG}
  \item \bibentry{Kiar2020instabilities}
  \item \bibentry{DeLuca2021Memento}
  \item \bibentry{Kruper2021evaluating}
  \item \bibentry{henriques2021dki}
  \item \bibentry{richford2021sgl}
  \item \bibentry{cieslak2021qsiprep}
  \item \bibentry{Gau2021-mh}
  \item \bibentry{mehta2021glaucoma}
  \item \bibentry{richie-halford-groupyr}
  \item \bibentry{rokem2020gigascience}
  \item \bibentry{chandio2020buan}
  \item \bibentry{bressler2020slow}
  \item \bibentry{beyeler2019miccai}
  \item \bibentry{bain2019laterality}
  \item \bibentry{beyeler2019shapes}
  \item \bibentry{lee2019perfusion}
  \item \bibentry{keshavan2019braindr}
  \item \bibentry{curtis2019joss}
  \item \bibentry{Huber2019DevSci}
  \item \bibentry{tian2019generalized}
  \item \bibentry{Smith2018JOSS}
  \item \bibentry{huppenkothen2018}
  \item \bibentry{Huber2018-ko}
  \item \bibentry{adam_richie-halford-proc-scipy-2018}
  \item \bibentry{Rokem2018-on}
  \item \bibentry{Yeatman2018-ug}
  \item \bibentry{xiao2017oir}
  \item \bibentry{polimis2017confidence}
  \item \bibentry{beyeler2017scipy}
  \item \bibentry{lee2017deep}
  \item \bibentry{mehta2017comparative}
  \item \bibentry{holdgraf2017portable}
  \item \bibentry{beyeler2017learning}
  \item \bibentry{henriques2017re}
  \item \bibentry{Ferizi2017diffusion}
  \item \bibentry{Rokem2017JoVWM}
  \item \bibentry{DeSimone2016}
  \item \bibentry{Cameron_Craddock2016-wc}
  \item \bibentry{Gorgolewski2016bids}
  \item \bibentry{MezerCoils}
  \item \bibentry{Tian2016QSpace}
  \item \bibentry{AjinaBlindsight}
  \item \bibentry{RokemDSSG}
  \item \bibentry{Rokem2015PLoS}
  \item \bibentry{Takemura2015CerCor}
  \item \bibentry{Zheng2014NIPS}
  \item \bibentry{Yeatman2014PNAS}
  \item \bibentry{Pestilli2014NatMeth}
  \item \bibentry{Garyfallidis2014FrontNeuroinf}
  \item \bibentry{McDevitt2014VisRes}
  \item \bibentry{Kay2013FrontNeurosci}
  \item \bibentry{Yoon2013FrontPsychol}
  \item \bibentry{Rokem2013FrontCompNeurosci}
  \item \bibentry{Kay2013PLoSCompBiol}
  \item \bibentry{Kosovicheva2012FrontBehavNeurosci}
  \item \bibentry{Rokem2012CerCor}
  \item \bibentry{Rokem2011FrontHumNeurosci}
  \item \bibentry{Rokem2010CurrBiol}
  \item \bibentry{Rokem2010Neuropsychpharmacology}
  \item \bibentry{Yoon2010JNeurosci}
  \item \bibentry{Eyherabide2009FrontNeurosci}
  \item \bibentry{Rokem2009Scipy}
  \item \bibentry{Yoon2009SchizBull}
  \item \bibentry{Rokem2009BrainRes}
  \item \bibentry{Rokem2009Neuropsych}
  \item \bibentry{Eyherabide2008FrontCompNeurosci}
  \item \bibentry{Rokem2006JNeurophysiol}
\end{etaremune}

\textbf{* indicates equal contribution}

\section*{Books and book chapters}
  \begin{etaremune}
\item \bibentry{rokem2023nds}
\item \bibentry{rokem2018repro_case_study}
\item \bibentry{rokem2018repro_glossary}
\item \bibentry{rokem2018repro_assessment}
  \end{etaremune}

\section*{White papers and work in progress}
\begin{enumerate}
  \item \bibentry{RokemBenson2024NeuroHackademy}
  \item \bibentry{Roy2023SEDA}
  \item \bibentry{Rokem2023cnn}
  \item \bibentry{pogoncheff2023prostheses}
  \item \bibentry{Kruper2023Glaucoma}
  \item \bibentry{rokem_dichter_holdgraf_ghosh_2021}
\end{enumerate}

\textbf{* indicates equal contribution}


\section*{Online courses}
\begin{enumerate}
  \item \bibentry{rokem2018datacamp_cnn}
  \item \bibentry{rokem2019datacamp_mpl}
\end{enumerate}

\section*{Honors and awards}

\years{2023} McGill University Neuro / Irv and Helga Cooper Foundation Open Science Prize for international projects, received as a member of the Brain Imaging Data Structure steering group (\$ 80,000 CAD).\\
\years{2022 -- 2024} Elected member of the Brain Imaging Data Structure (BIDS) steering group. \\
\years{2012 -- 2015} NIH Postdoctoral National Research Service Award. \\
\years{2009 -- 2010} NIH Predoctoral National Research Service Award. \\



\section*{Funded research}
\years{2023-2024} NSF Workshop: Towards an Open Source Model for Data and Metadata Standards (role: PI) \$ 99,953.00\\
\years{2018-2027} NIH R01: Community-supported open-source software for computational neuroanatomy (role: subcontract PI, PI: Eleftherios Garyfallidis), \$ 2,726,578 \\
\years{2019-2024} NIH BRAIN Initiative RF1: A data science toolbox for analysis of Human Connectome Project diffusion MRI (role: PI), \$707,444.\\
\years{2017-2027} NIH/National Institute for Mental Health, R25: Summer Institute in Neuroimaging and Data Science (role: PI), \$2,003,598\\
\years{2021-2026} NIH U19: Adult Changes in Thought (ACT) Research Program (role: Senior Personnel, PI: Eric Larson and Paul Crane), \$23,352,014\\
\years{2022-2024} Chan Zuckerberg Initiative Essential Open Source Software: Diffusion Imaging in Python (role: co-Investigator; PI: Serge Koudoro), \$165,407\\
\years {2021-2024} NIH R01 A community-driven development of the brain imaging data standard (BIDS) to describe macroscopic brain connections (role: Senior Personnel; PI: Franco Pestilli) \$352,342. \\
\years{2021-2024} NIH R01 NIPreps: integrating neuroimaging preprocessing workflows across modalities, populations, and species (role: multi-PI) \$1,630,690 \\
\years{2019-2024} NIH R01: Aging eyes and aging brains in studying Alzheimer's disease: modern ophthalmic data collection in the Adult Changes in Thought (ACT) study (role: Senior Personnel; PI: Cecilia Lee). \$17,197,690  \\

\section*{Completed funded research}
\years{2019-2023} NSF BDHUBS: Collaborative Proposal: West: Accelerating the Big Data Innovation Ecosystem (role: Senior Personnel, PI: Ed Lazowska), \$201,822.\\
\years{2019-2023} NSF HDR: I-DIRSE-FW: Accelerating the Engineering Design and Manufacturing Life-Cycle with Data Science (role: co-PI; PI: Magda Balazinska), \$2,320,979. \\
\years{2017-2023} The Bill \& Melinda Gates Foundation: Advance Data Analytic Support for Strategic PNW Partners by eScience Institute (role: co-PI, PI: Bill Howe), \$754,601. \\
\years{2021-2022} UW Azure Cloud Computing Credits (role: PI), \$43,000\\
\years{2021-2022} NSF AccelNet Exchange Grant through the International Network for Biologically-Inspired Computing \$10,000.\\
\years{2018-2022} NIH BRAIN Initiative U19: Computational and Circuit Mechanisms Underlying Rapid Learning. (role: Data Science Core Senior Personnel, PI: Beth Buffalo), \$14,439,172. \\
\years{2021} Google Cloud Research Credits (role: PI), \$5,000\\
\years{2021} Amazon Web Services Cloud Computing Credits (role: PI), \$5,000 \\
\years{2020} Google Cloud Research Credits (role: PI), \$5,000\\
\years{2021} UW Azure Cloud Computing Credits (role: PI), \$20,000\\
\years{2020} Google Cloud Research Credits (role: PI), \$5,000\\
\years{2018 - 2021} NSF TRIPODS + X EDU: Foundational training in neuroscience and geoscience via hack weeks (role: co-PI, PI: Maryam Fazel), \$ 185,058. \\
\years{2017-2020} NSF SI2-SSE: An ecosystem of reusable image analytics pipelines (role: co-PI, PI: Andy Connoly), \$500,000. \\
\years{2019} Google TensorFlow Research Cloud credits, 100 TPU hours {role: PI}.\\
\years{2017-2018} The Bill \& Melinda Gates Foundation: The King County Analytics Project (role: co-PI, PI: Bryna Hazelton), \$320,000. \\
\years{2017-2018} NSF ACI SI2-S2I2: Conceptualization: Conceptualizing a US Research Software Sustainability Institute (URSSI) (role: senior Personnel, PI: Daniel Katz), \$ 499,999.\\
\years{2015-2018} NSF BDHUBS: A Big Data Innovation Hub for the Western United States (role: co-PI, PI: Ed Lazowska), \$201,822.\\
\years{2017} XSEDE: Educational allocation for a one-day course in neuroscience and data science (role: PI), 10,000 core hours. \\
\years{2016} Amazon Web Services cloud computing credits for research (role: PI), \$30,000. \\
\years{2015-2016} The Bill \& Melinda Gates Foundation: Increasing Data-Driven Decision Making through Data Modeling Techniques and Best Practices (role: co-PI, PI: Bryna Hazelton), \$140,995. \\
\years{2012} Stanford CNI Seed grant: Diffusion MRI measured with multiple  b-values (role: PI), \$8,000. \\
\years{2012-2015} NIH/National Eye Institute, National Research Service Award (F32): The Anatomical Basis of Texture Perception in Central and Peripheral Visual Field (role: PI),  \$155,346. \\
\years{2009-2010} NIH/National Institute for Aging, National Research Service Award (F31): Neural Mechanisms of Perceptual Learning (role: PI), \$22,253. \\

  \section*{Invited talks}
  \years{11/2023} Tanenbaum Open Science Institute Leaders Council. \\
  \years{11/2023} McGill University Quantitative Life Sciences and Medicine seminar series.  \\
  \years{4/2023} Academic Data Science Alliance. ``Careers in neuroscience and data science'' panel. \\
  \years{11/2022} Society for Neuroscience Professional Development Workshop: ``Brain Data Science: A World of New Neuroscience Career Opportunities''.\\
  \years{7/2022} Invited talk at Pacific Northwest National Lab MARS Seminar (online).\\
  \years{8/2021} Invited talk at  Research Running on Cloud Compute \& Emerging Technologies (RRoCCET) 2021 (online). \\
  \years{6/2021} Oregon State University (Corvalis) and Nanyang Techincal University (Singapore) CN Yang Scholars program (online) \\
  \years{10/2020} Open Data Science Conference West, San Francisco, CA (and online). \\
  \years{9/2020} Amazon Web Services Education: Research Seminar Series (online) \\
  \years{9/2019} INCF Neuroinformatics congress, Warsaw, Poland (Keynote). \\
  \years{5/2019} Halıcıoğlu Data Science Institute, University of California, San Diego.\\
  \years{5/2019} Northwest Data Science Summit, University of Washington, Seattle, WA. \\
  \years{10/2018} Carnegie Mellon University, Open Science Symposium, Pittsburgh, PA. \\
  \years{10/2018} Presentation to the Advisory Council to the NIH Director Working Group for the BRAIN Initiative 2.0, Baylor College of Medicine, Houston, TX.\\
  \years{6/2018} ISMRM educational course: "Modeling diffusion MRI", Paris, France. \\
  \years{5/2018} Edmund and Lily Safra Center for Brain Science, The Hebrew University of Jerusalem, Israel.\\
  \years{5/2018} Department of Physiology and Biophysics, University of Washington, Seattle, WA. \\
   \years{11/2017} Center for Studies in Demography and Ecology, University of Washington, Seattle, WA.\\
   \years{7/2017} PNW Prostate Cancer SPORE annual meeting, Seattle, WA.\\
   \years{11/2015} Psychology Department, Indiana University, Bloomington, IN. \\
   \years{3/2015} Berkeley Institute for Data Science. Berkeley, CA\\
   \years{8/2014} Neuroimaging Laboratory, Washington University, Saint Louis, MO.\\
   \years{5/2013} Max Planck Institute for Brain Research, Frankfurt, Germany.\\
   \years{1/2012} Tech talk at GitHub Inc, San Francisco, CA \\
   \years{6/2011} Psychology Department, Dartmouth University. Hannover, NH.\\
   \years{1/2011} Center for Magnetic Resonance Research, University of Minneapolis. Twin Cities, MN\\
   \years{1/2011} Department of Psychology, Vanderbilt University. Nashville, TN.\\
   \years{3/2009} Posit Science. San Francisco, CA.\\
   \years{10/2008} Stanford Vision Lunch. Stanford, CA.\\
   \years{9/2008} The Institute for Theoretical Biology, Humboldt University, Berlin, Germany.\\
   \years{9/2008} The Institute for Biology, Ludwig-Maximillian University, Munich, Germany.\\
   \years{8/2008} UC Davis, Imaging Research Center. Davis, CA.\\

\section*{Software}

\subsection*{Core contributions}

\years{2015 -- } \emph{pyAFQ}: automated quantification of brain white matter fibers \url{https://yeatmanlab.github.io/pyAFQ/}.

\years{2017 -- } \emph{AFQ-Browser} \url{https://yeatmanlab.github.io/AFQ-Browser}.

\years{2020 -- } \emph{AFQ-Insight}: Statistical learning for tractometry data  \url{https://yeatmanlab.github.io/AFQ-Insight/}.

\years{2022 -- } \emph{TractR}: Statistical learning for tractometry data  \url{https://github.com/yeatmanlab/tractr/}.

\years{2017 -- } \emph{Cloudknot}: a pythonic interface to AWS Batch Services. \url{https://nrdg.github.io/cloudknot}

\years{2016 -- } \emph{Pulse2percept}: Models for Sight Restoration. \url{https://uwescience.github.io/pulse2percept/}

\years{2011 -- } \emph{DIPY}: diffusion MRI in Python, \url{http://dipy.org}

\years{2008 -- } \emph{Nitime}: Time-series analysis for neuroscience, \url{http://nitime.org}

\subsection*{Minor contributions}

Minor contributions across many open source software libraries in the Python
scientific eco system, including \emph{Scipy}, \emph{Matplotlib}, \emph{Scikit
Learn}, \emph{Scikit Image}, \emph{Jupyter} and \emph{IPython}, as well as many
neuroscience-specific software libraries, including \emph{Nibabel},
\emph{Nipype}, \emph{Nipy}. Full record of open-source software contributions available at \url{https://github.com/arokem}

\section*{Data sets}
\years{2013} Human brain diffusion-weighted MRI, collected with high diffusion-weighting angular resolution and repeated measurements at multiple diffusion-weighting strengths, \url{https://purl.stanford.edu/ng782rw8378}\\
\years{2012} Test-retest Diffusion MRI, measured at 1.5 mm isotropic resolution, b-value=2000 $s/mm^2$ , \url{https://purl.stanford.edu/rt034xr8593}\\
\years{2005} Intracellular recordings from insect primary auditory receptor neurons, \url{https://crcns.org/data-sets/ia/ia-1}\\

\section*{Selected Conference presentations}
\begin{enumerate}

\item Early life adversity and white matter development. Adam Richie-Halford, Ethan Roy, John Kruper, Jason Yeatman, {\bf Ariel Rokem}. Annual Meeting of the Society for Neuroscience, 2022.

\item Deep learning for analysis of diffusion-MRI based white matter tractometry. Joanna Qiao, Jason Yeatman, {\bf Ariel Rokem}, Adam Richie-Halford (2022). Annual Meeting of the Society for Neuroscience, 2022.

\item Francois Rheault, Val\'{e}rie Hayot-Sasson, Robert E. Smith, Christopher Rorden, Jacques-Donald Tournier, Eleftherios Garyfallidis, Fang-Cheng Yeh, Christopher J. Markiewicz, Matthew Brett, Ben Jeurissen, Paul A. Taylor, D. Baran Aydogan, Derek A. Pisner, Serge Koudoro, Soichi Hayashi, Daniel Haehn, Steve Pieper, Daniel Bullock, Emanuele Olivetti, Jean-Christophe Houde, Marc-Alexandre C\^{o}t\'{e}, Flavio Dell’Acqua, Alexander Leemans, Maxime Descoteaux, Bennett Landman, Franco Pestilli, and {\bf Ariel Rokem} (2002). TRX: A community-oriented tractography file format. Annual Meeting of the Organization for Human Brain Mapping, 2022.

\item Adam Richie-Halford, Matthew Cieslak, Azeez Adebimpe, Sydney Covitz, McKenzie Paige Hagen, John Kruper, Mengjia Lyu, Oscar Miranda-Dominguez, Audrey Houghton, Damien Fair, Jason D. Yeatman, Theodore D. Satterthwaite, {\bf Ariel Rokem}. (2022) NIRV: The NeuroImaging Report Viewer. Annual Meeting of the Organization for Human Brain Mapping, 2022.

\item Mareike Grotheer, David Bloom, John Kruper, Manjari Narayan, Adam Richie-Halford, Vicente A. Aguilera González, Jason D. Yeatman, Kalanit Grill-Spector, and {\bf Ariel Rokem} (2022) Spatiotemporal differences in development of preterm infants white matter bundles are explained by faster \emph{in utero} compared to \emph{ex utero} myelination. Annual Meeting of the Organization for Human Brain Mapping, 2022.

\item Manjari Narayan, Noah Simon, Adam Richie-Halford, Jason Yeatman, {\bf Ariel Rokem} (2021). Nonparametric causal analysis of brain and cognition, applied to developmental neuroimaging. Annual Meeting of the Organization for Human Brain Mapping 2021.

\item John Kruper, Jason D. Yeatman, Adam Richie-Halford,David Bloom, Mareike Grotheer, Sendy Caffarra, Gregory Kiar, Iliana I. Karipidis, Ethan Roy, {\bf Ariel Rokem} (2021). Evaluating the reliability of diffusion-MRI based tractometry. Annual Meeting of the Organization for Human Brain Mapping 2021.

\item Adam Richie-Halford, Matthew Cieslak, Alexandre R. Franco, Valerie J. Sydnor, Jason Yeatman, Lei Ai, Michael Milham, Theodore D. Satterthwaite, {\bf Ariel Rokem} (2021). A preprocessed open diffusion derivatives dataset from the Healthy Brain Network. Annual Meeting of the Organization for Human Brain Mapping 2021. Received \emph{Merit Abstract Award}.

\item Mauro Bisson, Josh Romero, Thorsten Kurth, Massimiliano Fatica, Pablo F. Damasceno, Xihe Xie, Adam Richie-Halford, Serge Koudoro, Eleftherios Garyfallidis, {\bf Ariel Rokem} (2021). GPU-accelerated diffusion MRI tractography in DIPY. International Society for Magnetic Resonance in Medicine 2021

\item Rafael Neto Henriques, Marta Correia, Maurizio Marrale, Elizabeth Huber, John Kruper, Serge Koudoro, Jason Yeatman, Eleftherios Garyfallidis, {\bf Ariel Rokem} (2021). Diffusional Kurtosis Imaging in the Diffusion Imaging in Python Project. Inernational Society for Magnetic Resonance in Medicine 2021

\item A. Richie-Halford, J. Yeatman, N. Simon, and {\bf A. Rokem} (2021). Multidimensional analysis and detection of informative features in diffusion MRI measurements of human white matter. Inernational Society for Magnetic Resonance in Medicine 2021. Received the \emph{Magna Cum Laude} award based on reviewer scores.

\item A. Keshavan, J. Yeatman, {\bf A. Rokem} (2019). Swipes for science: An
open-source gamified citizen science framework for scalable data annotation.
Organization for Human Brain Mapping, 2019.

\item A. Richie-Halford J. Yeatman, {\bf A. Rokem}, A. Keshavan (2019).
DMRIprep: a Robust, Scalable Preprocessing Pipeline for diffusion MRI.
Organization for Human Brain Mapping 2019.

\item S. Xiao, Y. Wu, A. Y. Lee, {\bf A. Rokem} (2019). MRI2MRI: deep learning
neural networks infer brain diffusion properties from T1-weighted MRI. Organization for Human Brain Mapping 2019.

\item A. Richie-Halford, Jason Yeatman, Noah Simon, and {\bf A. Rokem} (2018, 2019). Multidimensional analysis and detection of informative features in diffusion MRI measurements of human white matter. Society for Neuroscience, 2018. Organization for Human Brain Mapping 2019.

\item S. Xiao, Y. Wu and A.Y. Lee and {\bf A. Rokem} (2018).
MRI2MRI: A deep convolutional network that accurately transforms between brain MRI contrasts.
International Society for Magnetic Resonance in Medicine, 2018

\item Q. Tian, G. Yang, C.W.U. Leuze, {\bf A. Rokem}, B.L. Edlow, J. McNab
(2017). Model-free Fourier Reconstruction of Diffusion Propagator and
Orientation from Multi-b-shell Diffusion MRI Data. Annual Meeting of the The
International Society for Magnetic Resonance in Medicine, 2017 and Annual Meeting of the Organization of Human Brain Mapping, 2017.

\item R. Neto-Henriques, Ø. Bergmann, {\bf A. Rokem}, O. Pasternak, M. M.
Correia (2017).  Exploring the potentials and limitations of improved free-water
elimination DTI techniques (2017). Annual Meeting of the The International
Society for Magnetic Resonance in Medicine, 2017.

\item M. Beyeler, {\bf A. Rokem}, G.M. Boynton, I. Fine (2017).
Reverse-engineering optimized stimulation protocols in epiretinal prosthesis
patients. The Eye and the Chip conference, 2017

\item M. Beyeler, {\bf A. Rokem}, G.M. Boynton, I. Fine (2017). Modeling
perceptual experience of retinal prosthesis patients during paired-electrode
stimulation. Cosyne 2017.

\item {\bf A. Rokem}, L. Huber, P. Mehta, R. Henriques, M. Balazinska, J. Yeatman (2016). Diffusion Kurtosis Imaging for the Human Connectome Project. Annual meeting of the Society for Neuroscience, San Diego, CA.

\item {\bf A. Rokem}. Future Proofing Data Intensive Research at the University of Washington eScience Institute. Talk at the UW IT Tech Connect Conference, March 2016. Slides: \url{http://arokem.github.io/2016-03-24-techconnect/}

\item {\bf A. Rokem}, E. Garyfallidis, F. Pestilli, B. Wandell, Statistical learning in DIPY. Talk presented at the 2015 Scientitfic Computing in Python meeting, Austin, TX and at PyData NW, Redmond, WA. Slides: \url{http://arokem.github.io/2015-pydatanw/}

\item S. Ogawa, H. Takemura, M. Terao, T. Haji, {\bf A. Rokem}, F. Pestilli, J.D. Yeatman, H. Horiguchi, H. Tsuneoka, B.A. Wandell, Y. Masuda(2014) Trans-synaptic changes in central white matter pathways in retinitis pigmentosa. Annual meeting of the Society for Neuroscience, Washington D.C.

\item {\bf A. Rokem}, G.S. Tang, T. Lucas, A. Thamrongrattanarit, L. Baltusis, R.F. Dougherty, R. Mata, L.L. Carstensen, G.R. Samanez-Larkin, and S.M. McClure (2014). Exploration and exploitation in action selection in humans depends on striatal GABA. Annual meeting of the Society for Neuroscience, Washington D.C.

\item {\bf A. Rokem} and F. Pestilli (2014). Measuring and modeling diffusion and white matter tracts. Symposium talk at the Vision Science Society meeting. St Pete's Beach, FL.

\item {\bf A. Rokem}, K. L. Chan, J.D. Yeatman, F. Pestilli, A. Mezer, and B. A. Wandell (2014). Evaluating the accuracy of diffusion models at multiple b-values with cross-validation. Annual meeting of the Society for Magnetic Resonance in Medicine. Milan, Italy.

\item Q. Tian, {\bf A. Rokem}, B. L. Edlow, R. Folkerth, and J. A. McNab (2014). Aliasing Artifacts in Orientation Distribution Functions: A Diffusion Spectrum Imaging Study. Annual meeting of the Society for Magnetic Resonance in Medicine. Milan, Italy.

\item {\bf A. Rokem} (2013). Tools for reproducible neuroimaging: an example from diffusion MRI. eResearch NZ, Christchurch, 2013.

\item {\bf A. Rokem}, J.D. Yeatman, F. Pestilli, K.N. Kay, A. Mezer, S. Van der Walt and B.A. Wandell (2013). Evaluating models of diffusion MRI data with cross-validation. Annual meeting of the Organization for Human Brain Mapping, Seatlle, WA.

\item F. Pestilli, J.D. Yeatman, {\bf A. Rokem}, K.N. Kay and B.A. Wandell (2013). Statistical evaluation of white matter connections. Annual meeting of the International Society for Magnetic Resonance in Medicine, Salt Lake City, UT and annual meeting of the Organization for Human Brain Mapping, Seattle, WA.

\item H. Takemura, F. Pestilli, {\bf A. Rokem}, J. Winawer, J.D. Yeatman and B.A. Wandell (2013). The visual dorsal and ventral streams communicate through the vertical occipital fasciculus. Annual meeting of the Organization for Human Brain Mapping, Seattle, WA.

\item A. Mezer, J.D. Yeatman, {\bf A. Rokem} and B.A. Wandell (2013). Language white matter tract laterality: from tractography to biophysical meaning. Annual meeting of the Organization for Human Brain Mapping, Seattle, WA.

\item J.D. Yeatman, A. Mezer, {\bf A. Rokem}, F. Pestilli, H. Feldman and B.A. Wandell (2013) Automated Fiber-tract Quantification of White Matter Tissue Biology. Annual meeting of the Organization for Human Brain Mapping, Seattle, WA.

\item {\bf A. Rokem}, and Landau A.N. (2013). Voluntary attention does not alleviate orientation specific surround suppression. Vision Sciences Society annual meeting. Naples, FL.

\item {\bf A. Rokem}, and M.A. Silver Cholinergic enhancement increases information content of stimulus representations in human visual cortex (2012). Presentation at the symposium: Neuromodulatory Mechanisms, New Orleans, LA.

\item E. McDevitt, B. Bays, {\bf A. Rokem}, M. A. Silver and S.C. Mednick (2012). Men need a nap to show perceptual learning of motion direction discrimination, but women do not. Vision Sciences Society Meeting, Naples, FL.

\item {\bf A. Rokem}, Michael A. Silver (2012). Cholinergic enhancement of perceptual learning in the human visual system. Oral presentation at a symposium on neuromodulation of visual perception. Vision Sciences Society Meeting, Naples, FL.

\item {\bf A. Rokem}, Michael A. Silver, Elizabeth A. McDevitt and Sara C. Mednick (2011), The effects of naps on the magnitude and specificity of perceptual learning of motion direction discrimination. Vision Sciences Society Meeting, Naples, FL.

\item {\bf A. Rokem}, R.E. Ooms, J.H. Yoon, M.J. Minzenberg, C.S. Carter and M.A. Silver (2010), Broader tuning for stimulus orientation in patients with schizophrenia. Annual Meeting of the Society for Neuroscience, San Diego, CA.

\item {\bf A. Rokem}, F. Perez, M. Trumpis, P. Ivanov, K. Koepsell, T. Blanche, D. Fegen, M. D'Esposito (2010). Nitime: an open-source library for time-series analysis of neuroscience data. Annual meeting of the Organization for Human Brain Mapping, Barcelona, Spain.

\item W. Prinzmetal, {\bf A. Rokem}, A.N. Landau, D. Wallace, M.A. Silver, M. D’Esposito (2010). The effects of the D2 dopamine receptor agonist bromocriptine on voluntary and involuntary spatial attention in humans. Vision Sciences Society Meeting, Naples FL.

\item {\bf A. Rokem} and M.A. Silver (2010) Cholinergic enhancement augments perceptual learning in the human visual system: a pharmacological fMRI study. Vision Sciences Society Meeting, Naples, FL.

\item {\bf A. Rokem} and F. Perez (2009). Time-series analysis in Nipy. The 8th Python in Science Conference (SciPy 2009), Pasadena, CA

\item {\bf A. Rokem}, D. Garg, A. Landau, W. Prinzmetal and M.A. Silver (2009). Effects of cholinergic enhancement on voluntary and involuntary attention. Vision Science Society Meeting, Naples, FL, Annual Meeting of the Society for Neuroscience, Chicago, IL and CSAIL conference, Hood River, OR.

\item D.W. Bressler, {\bf A. Rokem}, M.A. Silver (2009). Visual spatial attention improves fMRI response reliability by decreasing the amplitude of endogenous slow oscillations in visual cortex. Annual Meeting of the Society for Neuroscience, Chicago, IL.

\item {\bf A. Rokem} and M.A. Silver (2008) Cholinergic enhancement augments perceptual learning in the human visual system. Annual Meeting of the Society for Neuroscience, Washington, DC.

\item M.A. Silver, J. Yoon, {\bf A. Rokem}, M.J. Minzenberg and C.S. Carter (2008) Reduced orientation-specific surround suppression in schizophrenia. Annual Meeting of the Society for Neuroscience, Washington, DC.

\item J.-H. Schleimer, {\bf A. Rokem}, M.B. Stemmler (2008) Optimal phase dynamics of oscillatory neurons, the spike-triggered stimulus covariance, and maximal information transfer. Annual Meeting of the Society for Neuroscience, Washington, DC.

\item {\bf A. Rokem}, S. Sanghvi and M. Silver (2007). Motion adaptation bandwidth anisotropies in the human visual system. The Optical Society of America Fall Vision Meeting, Berkeley, CA, September 2007 and Dynamical Neuroscience XV – 3rd Annual Computational Cognitive Neuroscience Conference, San Diego, CA, November 2007 (selected to appear in a special issue of Brain Research devoted to Computational Cognitive Neuroscience).

\item I. Samengo, H.G. Eyherabide, {\bf A. Rokem} and A.V.M. Herz (2006). Information transmission in burst spiking. 2nd Bernstein Symposium for Computational Neuroscience. Berlin, Germany, 2006

\item M. Nahum, {\bf A. Rokem}, I. Nelken and M. Ahissar (2004). Speech intelligibility \& binaural interactions: effects of stimulus familiarity, stimulus similarity \& set size. The annual meeting of the Israeli Society for Neuroscience, the 30th Goettingen Neurobiology conference, and Cosyne 2005.

\item {\bf A. Rokem} and M. Ahissar (2004). Interactions between sensory and cognitive abilities in early-blind individuals. The annual meeting of the Israeli Society for Neuroscience and at the 30th Goettingen Neurobiology conference.

\item S. Watzl, {\bf A. Rokem}, T. Gollisch and A.V. Herz (2003). Coding capacitites of auditory receptor cells under different stimulus conditions. The 29th Goettingen Neurobiology conference.
\end{enumerate}


\section*{Teaching}

\subsection*{Classes}
\years{Spring 2024} Informatics for Psychology

\years{5/2021} Guest instructor -- Image Analysis for Data Scientists,  UW Department of Chemical Engineering (Instructor: Chad Curtis)

\years{5/2018} Guest instructor -- Data Science and Society, UW Department of Sociology (Instructor: Afra Mashhadi)

\years{10/2017} Guest instructor -- Data Science and Society, UW Department of Sociology (Instructor: Afra Mashhadi)

\years{11/2015} Guest Instructor -- eScience Python Seminar (Instructor: Jake Vanderplas).

\years{4/2014} Guest Instructor: MA capstone class, Department of Statistics, University of California, Berkeley (Instructor: Victoria Stodden).

\years{10/2013} Guest Instructor -- MRI methods, Department of Psychology, Stanford University (Instructor: Brian Wandell).

\years{Spring 2008} Teaching Assistant -- Brain Mind and Behavior, Department of Biology, UC Berkeley (Instructor: David Presti).

\years{Fall 2006} Teaching Assistant -- Mammalian Neuroanatomy, Department of Biololgy, UC Berkeley (Instructor: Jeff Winer).

\years{Fall 2003} Teaching Assistant -- Perception, Department of Psychology, Hebrew University of Jerusalem (Instructor: Merav Ahissar).


\subsection*{Software and Data Carpentry}
\years{2022} University of Washington eScience Institute: led instruction of pilot workshop in Image Processing in Python, 20 participants.

\years{2015 -- 2020 } University of Washington eScience Institute: led instruction of 20 workshops, >1,000 participants from >30 departments on campus.

\years{2015 --} University of Washington eScience Institute: led 3 Carpentries instructor training workshops. Trained >40 Carpentries instructors.

\years{2018 -- 2021} Annual Instructor Training, The West Big Data Innovation Hub. Seattle, WA. Trained more than >50 Carpentries Instructors.

\years{4/2018} Southern California Tribal Digital Village, Pala, CA

\years{7/2016} Instructor Training, SciPy Annual Conference, Austin, TX.

\years{12/2016} Instructor Training, Pacific Northwest National Lab, Richland, WA.

\years{9/2016} Instructor Training, Oregon State University, Corvalis, OR.

\years{11/2015} Data Carpentry (neuroimaging), Indiana University Psychology Department.

\years{11/2014} Stanford University. Stanford, CA.

\years{8/2014}  Washington University, St. Louis, MO

\years{9/2013}  University of Southern California. Los Angeles, CA

\years{6/2013} Christchurch University. Christchurch, New Zealand.

\years{4/2013} Lawrence Berkeley National Lab. Berkeley, CA

\years{3/2013} Stanford University. Stanford, CA

\subsection*{Other workshops}

\years {11/2023} Faculty: African Brain Data Science Academy \url{https://africanbraindatanetwork.com/abds-academy/}.

\years{1/2022} Organizer and lead instructor: Workshop on data science training and collaboration in Hispanic-Serving Institutions (West Big Data Hub and HSI STEM Hub; \url{https://uwescience.github.io/dstc-20220118/}).

\years{6/2021} Organizer and lead instructor: Workshop on data science training and collaboration in Hispanic-Serving Institutions (West Big Data Hub and HSI STEM Hub; \url{https://uwescience.github.io/dstc-2021/}).

\years{2021 \& 2022} Co-organizer (with Catherine Lebel): ``Tractometry : peering into the white matter'', educational course at the annual meeting of the Organization for Human Brain Mapping

\years{9/2019} Organizer and lead instructor: workshop on data science training and collaboration in Hispanic-Serving Institutions (West Big Data Hub and HSI STEM Hub; \url{https://uwescience.github.io/2019-09-16-dstc-workshop/}).

\years{2019 \& 2020} Co-organizer (with Andrew Doyle): ``Deep Learning in Human Brain Mapping'', educational course at the annual meeting of the Organization for Human Brain Mapping

\years{5/2015} Brainhacking 101. Organization for Human Brain Mapping annual meeting.

\years{3/2014} Python for Neuroscience Workshop, University of Nottingham, UK.

\years{10/2012} Reproducible Research in Neuroimaging Workshop. Stanford Center for Cognitive and Neurobiological Imaging.

\years{8/2007} Matlab and the Psychophysics Toolbox, Department of Psychology, UC Berkeley.

\section*{Mentorship}

\subsection*{Postdocs}

\years{2023 -- } Kelly Chang

\years{2020 -- 2022} Adam Richie-Halford. Currently Research and Development Scientist at Stanford.

\years{2020 -- 2021} Manjari Narayan (with Jason Yeatman). Currently Machine Learning Scientist at Dyno Therapeutics.

\years{2016 -- 2019} Michael Beyeler (with Ione Fine). Currently Assistant
Professor at the University of California, Santa-Barbara.

\years{2017 -- 2018} Anisha Keshavan (with Jason Yeatman). Currently Senior Data Scientist at Octave Biosciences.

\years{2016 -- 2017} Dongfang Zhao (With Magda Balazinska). Currently Assistant Professor at University of Nevada, Reno.

\subsection*{PhD students}

\subsubsection*{As principal advisor}

\years{2022 -- } John Kruper

\years{2021 -- } McKenzie Hagen

\subsubsection*{As secondary advisor}

\years{2017 -- } Ezgi Y\"{u}cel (with Ione Fine)

\years{2022 -- } Vaishnavi Mohan (with Ione Fine)

\subsection*{Post-baccalaureate students}

\years{2022 -- 2023}  Teresa Gomez (Research fellowship to increase diversity through administrative supplement to Nipreps grant)

\years{2020 -- 2022} John Kruper (UWIN post-baccalaureate fellow).

\years{2020 -- 2021} David Bloom.

\subsection*{PhD committee service}

\years{2021 -- 2023} Shervin Sahba, UW Physics

\years{2022} Kelly Chang, UW Department of Psychology

\years{2019 -- 2020} Parmita Mehta, UW Computer Science and Engineering.

\years{2017 -- 2020} Chad Curtis, UW Department of Chemical Engineering.

\years{2017 -- 2017} Kivan Polimis, UW Department of Sociology.

\subsection*{Google Summer of Code Open Source Software Interns}

\years{Summer 2016} Shahnawaz Ahmed (DIPY).

\years{Summer 2015} Rafael Neto-Henriques (DIPY).

\subsection*{Research interns and undergraduate students}

\years{2023} Qiqi Liang (Biology undergraduate student), Isaac Crane (Highschool student intern).

\years{2022} Joanna Qiao (Psychology independent study).

\years{2021} Leqi Teng (Psychology Honors Student), Cecilia Barnes (HCDE independent study).

\section*{Public Outreach}

\years{2023 -- } Lectures about early life brain development to small groups of parents through the Program for Early Parent Support (PEPS) Seattle.

\section*{Service}

\years{2021 -- } Advisory Committee for USC-based Reproducible Rehabilitation (ReproRehab) research education program \url{https://www.reprorehab.usc.edu/}

\years{2020 -- } Chair, International Neuroinformatics Coordinating Facility Training and Education Committee.

\years{2021} Chair, ``Data Science and Neuroinformatics'' symposium at the
International Neuroinformatics Coordination Facility Assembly

\years{2018 -- 2022} Member, Organization for Human Brain Mapping Education Committee.

\years{2017 -- 2020} Chair, University of Washington eScience Institute Special Interest Group on Neuroinformatics.

\years{2020 -- 2022} Deputy Chair, International Neuroinformatics Coordinating Facility Training and Education Committee.

\years{2017 --} Member of the International Neuroinformatics Coordinating Facility Training and Education Committee.

\years{2017 -- 2018} Chair, University of Washington eScience Institute Working Group on Reproducibility and Open Science.

\years{2017} Chair (with Olivia Guest) mini-symposium in neuroscience, \emph{Scientific Computing in Python} conference.

\years{2016 -- 2019} Co-PI of the Western Big Data Innovation Hub.

\years{2016 -- 2017} Organizer of the ImageXD workshop series on image processing across domains (\url{http://www.imagexd.org/}).

\years{2016 -- } Software Carpentry Instructor Trainer: training and certifying instructors for Software Carpentry.

\years{2016 --} Course Director: Summer Institute in Neuroimaging and Data Science (\url{https://neurohackademy.org}).

\years{2014} Chair (with Franco Pestilli): ``The White Matter Matters: Diffusion MRI in Vision Science''. Symposium at the Vision Sciences Society annual meeting.

\years{2012 --}  Software Carpentry Instructor.

\years{2007 - 2008} Coordinator, Working Group on Neuroscience and Philosophy of Mind, Townsend Center for the Humanities, University of California, Berkeley.

\vspace{4pt}

Editorial board member, \emph{Scientific Data} (2021 - ); Editorial board member, \emph{Journal of Open Source Software} (2016 - 2021);  Editorial board member, \emph{Journal of Open Research Software} (2016 - 2019); Associate Editor, \emph{Frontiers in Human Neuroscience} (2020 - ); Associate Editor \emph{Journal of Machine Learning Research} (2021- ). Review Editor for \emph{Proceedings of the National Academy of Science, USA} (2019). Editor, Special Topic: Explicating the interplay between anatomical and functional connectivity in the human brain, \emph{Frontiers in Human Neuroscience} (2015). Program committee member for \emph{Pattern Recognition in Neuroimaging} (2015, 2016), \emph{Scientific Computing in Python} (2016, 2017).

\vspace{4pt}

Reviewer for \emph{Annals of Applied Statistics}, \emph{PLoS One}, \emph{Human Brain Mapping}, \emph{Journal of Cognitive Neuroscience}, \emph{Frontiers in Human Neuroscience}, \emph{Frontiers in Abnormal Psychology}, \emph{Journal of Open Research Software}, \emph{Neuroimage}, \emph{Journal of Vision}, \emph{F1000 Research}, \emph{Journal of Neuroimaging}, \emph{Current Opinion in Neuroscience}, \emph{Psychophysiology}, \emph{Scientific Data}, \emph{Proceedings of the National Academy of Science, USA}, \emph{Neuroinformatics}, \emph{PLoS Computational Biology}, \emph{eLife}, \emph{IOVS}, \emph{Nature Communications}, \emph{Biological Psychiatry}, \emph{Imaging Neuroscience}, \emph{Ophthalmology Science}.

\vspace{4pt}

Grant reviewer for \emph{NIH}, \emph{NSF}, \emph{Academic Data Science Alliance}, \emph{Chan Zuckerberg Initiative}.


% Footer
\bigskip
\begin{center}
  \begin{footnotesize}
    Last updated: \today
  \end{footnotesize}
\end{center}

\nobibliography{arokemCV}
\end{document}
