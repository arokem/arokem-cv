
\documentclass[11pt,fullpage]{article}

\usepackage{cite}
\usepackage{bibentry}
\usepackage{hyperref}
\usepackage{geometry}
\usepackage{listliketab}
\usepackage{array}
\usepackage{longtable}
\usepackage{natbib}
\usepackage{fontspec}


\usepackage{marginnote} % For margin years
\newcommand{\years}[1]{\marginnote{\scriptsize #1}} % New command for including margin years
\renewcommand*{\raggedleftmarginnote}{}
\setlength{\marginparsep}{-1pt} % Slightly increase the distance of the margin years from the contant
\reversemarginpar

\setmainfont{Hoefler Text}
\nobibliography*
\bibliographystyle{apalike}

% Palatino font
%\usepackage[T1]{fontenc}
%\usepackage[sc,osf]{mathpazo}

\def\name{Ariel Rokem}

\reversemarginpar

\geometry{
  body={6.5in, 8.5in},
  left=1.0in,
  top=1.0in
}

% Customize page headers
\pagestyle{myheadings}
\markright{\name}
\thispagestyle{empty}

% Custom section fonts
\usepackage{sectsty}
\sectionfont{\rmfamily\mdseries\Large}
\subsectionfont{\rmfamily\mdseries\itshape\large}

% Other possible font commands include:
% \ttfamily for teletype,
% \sffamily for sans serif,
% \bfseries for bold,
% \scshape for small caps,
% \normalsize, \large, \Large, \LARGE sizes.

% Don't indent paragraphs.
\setlength\parindent{0em}

% Make lists without bullets
\renewenvironment{itemize}{
  \begin{list}{}{
    \setlength{\leftmargin}{1.5em}
  }
}{
  \end{list}
}

\hypersetup{linkcolor=blue,citecolor=blue,filecolor=black,urlcolor=MidnightBlue}

\begin{document}
%\bibliographystyle{apalike}

% Print name centered and bold:
\centerline{\Large \bf \name}

\vspace{0.25in}

\begin{minipage}{0.50\linewidth}
  WRF Data Science Studio\\
  Physics/Astronomy Tower, 6th Floor\\
  3910 15th Ave NE\\
  Campus Box 351570\\
  Seattle, WA 98105\\
\end{minipage}
\begin{minipage}{0.50\linewidth}
  \begin{tabular}{ll}
    Phone: & +1-510-3876264 \\
    Email: & \href{mailto:arokem@gmail.com}{arokem@gmail.com} \\
    Homepage: & \href{http://arokem.org/}{www.arokem.org} \\
    Orcid ID: & \href{http://orcid.org/0000-0003-0679-1985}{0000-0003-0679-1985} \\
  \end{tabular}
\end{minipage}

\section*{Work Experience}
\begin{tabular}{ll}
  2015 -  & {\bf University of Washington eScience Institute}\\
  & \emph{Senior Data Scientist}\\
  2011-2015 & {\bf Stanford University}\\
  & \emph{Postdoctoral Researcher}\\
  2010-2011 & {\bf University of California, Berkeley}\\
  & \emph{Postdoctoral Researcher}\\
  2002-2003 & {\bf Humboldt University, Berlin}\\
  & \emph{Research Student}\\

\end{tabular}

\section*{Education}

\begin{tabular}{ll}
	2010 & {\bf Ph.D. } Neuroscience, University of California, Berkeley \\
	2005 & {\bf M.A.} Cognitive Psychology, Hebrew University of Jerusalem \\
	2002 & {\bf B.Sc.} Biology and Psychology, Hebrew University of Jerusalem \\
\end{tabular}

\section*{Publications}

\setlength{\extrarowheight}{10pt}


  \years{Under review} \bibentry{huber2017readingwm}

  \vspace{4pt}

  \bibentry{huppenkothen2017hackweeks}

  \vspace{4pt}

  \bibentry{rokem2017resampling}

  \vspace{4pt}

  \bibentry{yeatman2017afqbrowser}

  \vspace{4pt}

  \bibentry{lee2017perfusion}

  \vspace{4pt}

  \bibentry{tian2017fouriereap}

  \years{2017}

  \bibentry{beyeler2017scipy}

  \vspace{4pt}

  \bibentry{lee2017deep}

  \vspace{4pt}

  \bibentry{holdgraf2017portable}

  \vspace{4pt}

  \bibentry{beyeler2017learning}

  \vspace{4pt}

  \bibentry{henriques2017re}

  \vspace{4pt}

  \bibentry{Ferizi2017diffusion}

  \vspace{4pt}

  \bibentry{Rokem2017JoVWM}

  \vspace{4pt}

  \years{2016}  \bibentry{DeSimone2016}

  \vspace{4pt}

  \bibentry{Cameron_Craddock2016-wc}

  \vspace{4pt}

  \bibentry{Gorgolewski2016bids}

  \vspace{4pt}

  \bibentry{MezerCoils}

  \vspace{4pt}

  \bibentry{Tian2016QSpace}

  \vspace{4pt}

  \years{2015}   \bibentry{AjinaBlindsight}

  \vspace{4pt}

  \bibentry{RokemDSSG}

  \vspace{4pt}

  \bibentry{Rokem2015PLoS}

  \vspace{4pt}

  \bibentry{Takemura2015CerCor}

  \vspace{4pt}

  \years{2014} \bibentry{Zheng2014NIPS}

  \vspace{4pt}

  \bibentry{Yeatman2014PNAS}

  \vspace{4pt}

  \bibentry{Pestilli2014NatMeth}

  \vspace{4pt}

  \bibentry{Garyfallidis2014FrontNeuroinf}

  \vspace{4pt}

  \bibentry{McDevitt2014VisRes}

  \vspace{4pt}

  \years{2013} \bibentry{Kay2013FrontNeurosci}

  \vspace{4pt}

  \bibentry{Yoon2013FrontPsychol}

  \vspace{4pt}

  \bibentry{Rokem2013FrontCompNeurosci}

  \vspace{4pt}

  \bibentry{Kay2013PLoSCompBiol}

  \vspace{4pt}

  \years{2012} \bibentry{Kosovicheva2012FrontBehavNeurosci}

  \vspace{4pt}

  \bibentry{Rokem2012CerCor}

  \vspace{4pt}

  \years{2011} \bibentry{Rokem2011FrontHumNeurosci}

  \vspace{4pt}

  \years{2010} \bibentry{Rokem2010CurrBiol}

  \vspace{4pt}

  \bibentry{Rokem2010Neuropsychpharmacology}

  \vspace{4pt}

  \bibentry{Yoon2010JNeurosci}

  \vspace{4pt}

  \years{2009} \bibentry{Eyherabide2009FrontNeurosci}

  \vspace{4pt}

  \bibentry{Rokem2009Scipy}

  \vspace{4pt}

  \bibentry{Yoon2009SchizBull}

  \vspace{4pt}

  \bibentry{Rokem2009BrainRes}

  \vspace{4pt}

  \bibentry{Rokem2009Neuropsych}

  \vspace{4pt}

  \years{2008} \bibentry{Eyherabide2008FrontCompNeurosci}

  \vspace{4pt}

  \years{2006} \bibentry{Rokem2006JNeurophysiol}


\section*{Funded research}
  \years{2017-2023} The Bill \& Melinda Gates Foundation: Advance Data Analytic Support for Strategic PNW Partners by eScience Institute (role: co-PI), \$750,000 \\
  \years{2017-2020} NSF SI2-SSE: An ecosystem of reusable image analytics pipelines (role: co-PI), \$500,000 \\
  \years{2017-2021} National Institute for Mental Health, R25: Summer Institute in Neuroimaging and Data Science (role: PI), \$974,253\\
  \years{2017-2018} The Bill \& Melinda Gates Foundation: The King County Analytics Project (role: co-PI), \$320,000 \\
  \years{2015-2016} The Bill \& Melinda Gates Foundation: Increasing Data-Driven Decision Making through Data Modeling Techniques and Best Practices (role: co-PI), \$140,995 \\
  \years{2015-2018} NSF BDHUB: A Big Data Innovation Hub for the Western United States (role: co-PI), \$201,822\\
  \years{2012} Stanford CNI: Seed grant for multi b-value DWI (role: PI), \$8,000 \\
  \years{2012-2015} National Eye Institute, F32: The Anatomical Basis of Texture Perception in Central and Peripheral Visual Field (role: PI),  \$155,346 \\
  \years{2009-2010} National Institute for Aging, F31: Neural Mechanisms of Perceptual Learning (role: PI), \$22,253 \\

  \section*{Invited talks}
   \years{6/2018} ISMRM educational course: "Modeling diffusion MRI", Paris, France \\
   \years{7/2017} PNW Prostate Cancer SPORE annual meeting, Seattle, WA \\
   \years{11/2015} Psychology Department, Indiana University, Bloomington \\
   \years{3/2015} Berkeley Institute for Data Science\\
   \years{8/2014} Neuroimaging Laboratory, Washington University, Saint Louis\\
   \years{5/2013} Max Planck Institute for Brain Research, Frankfurt\\
   \years{6/2011} Psychology Department, Dartmouth University.\\
   \years{1/2011} Center for Magnetic Resonance Research, University of Minneapolis\\
   \years{1/2011} Department of Psychology, Vanderbilt University.\\
   \years{3/2009} Posit Science.\\
   \years{10/2008} Stanford Vision Lunch.\\
   \years{9/2008} The Institute for Theoretical Biology, Humboldt University, Berlin\\
   \years{9/2008} The Institute for Biology, Ludwig-Maximillian University, Munich.\\
   \years{8/2008} UC Davis, Imaging Research Center.\\


\section*{Data sets}

\years{2013} Human brain diffusion-weighted MRI, collected with high diffusion-weighting angular resolution and repeated measurements at multiple diffusion-weighting strengths, \url{https://purl.stanford.edu/ng782rw8378}\\
\years{2012} Test-retest Diffusion MRI, measured at 1.5 mm isotropic resolution, b-value=2000 $s/mm^2$ , \url{https://purl.stanford.edu/rt034xr8593}\\
\years{2005} Intracellular recordings from insect primary auditory receptor neurons, \url{https://crcns.org/data-sets/ia/ia-1}\\

\section*{Software}

\subsection*{Core contributions}

These are software projects that I have led or to which I have contributed core contributions

\years{2017 -- } Popylar: Tracking software usage with Google Analytics. \url{https://popylar.github.io}
\years{2016 -- } Keratin: deep learning for biomedical imaging. \url{https://github.com/uw-biomedical-ml/keratin}
\years{2016 -- } Pulse2percept: Models for Sight Restoration. \url{https://uwescience.github.io/pulse2percept/}
\years{2015 --} Shablona: a template for small scientific Python projects. \url{https://github.com/uwescience/shablona}
\years{2013 -- } Popeye: A population receptive field estimation tool \url{https://kdesimone.github.io/popeye/}
\years{2011 -- 2015} Vistasoft: Matlab software for human MRI data analysis, \url{https://github.com/vistalab/vistasoft}\\
\years{2011 -- } Dipy: diffusion MRI in Python, \url{http://dipy.org}\\

\subsection*{Minor contributions}

Minor contributions across many open source software libraries in the Python
scientific eco system, including \emph{Matplotlib}, \emph{Scikit Learn},
\emph{Scikit Image}, \emph{Jupyter} and \emph{IPython}.

\section*{Selected Conference presentations}
\begin{enumerate}

\item Future Proofing Data Intensive Research at the University of Washington eScience Institute. Talk at the UW IT Tech Connect Conference, March 2016. Slides: \url{http://arokem.github.io/2016-03-24-techconnect/}

\item Statistical learning in DIPY. Talk presented at the 2015 Scientitfic Computing in Python meeting, Austin, TX and at PyData NW, Redmond, WA. Slides: \url{http://arokem.github.io/2015-pydatanw/}

\item S. Ogawa, H. Takemura, M. Terao, T. Haji, A. Rokem, F. Pestilli, J.D. Yeatman, H. Horiguchi, H. Tsuneoka, B.A. Wandell, Y. Masuda(2014) Trans-synaptic changes in central white matter pathways in retinitis pigmentosa. Annual meeting of the Society for Neuroscience, Washington D.C.

\item A. Rokem, G.S. Tang, T. Lucas, A. Thamrongrattanarit, L. Baltusis, R.F. Dougherty, R. Mata, L.L. Carstensen, G.R. Samanez-Larkin, and S.M. McClure (2014). Exploration and exploitation in action selection in humans depends on striatal GABA. Annual meeting of the Society for Neuroscience, Washington D.C.

\item A. Rokem and F. Pestilli (2014). Measuring and modeling diffusion and white matter tracts. Symposium talk at the Vision Science Society meeting. St Pete's Beach, FL.

\item A. Rokem, K. L. Chan, J.D. Yeatman, F. Pestilli, A. Mezer, and B. A. Wandell (2014). Evaluating the accuracy of diffusion models at multiple b-values with cross-validation. Annual meeting of the Society for Magnetic Resonance in Medicine. Milan, Italy.

\item Q. Tian, A. Rokem, B. L. Edlow, R. Folkerth, and J. A. McNab (2014). Aliasing Artifacts in Orientation Distribution Functions: A Diffusion Spectrum Imaging Study. Annual meeting of the Society for Magnetic Resonance in Medicine. Milan, Italy.

\item A. Rokem (2013). Tools for reproducible neuroimaging: an example from diffusion MRI. eResearch NZ, Christchurch, 2013.

\item A. Rokem, J.D. Yeatman, F. Pestilli, K.N. Kay, A. Mezer, S. Van der Walt and B.A. Wandell (2013). Evaluating models of diffusion MRI data with cross-validation. Annual meeting of the Organization for Human Brain Mapping, Seatlle, WA.

\item F. Pestilli, J.D. Yeatman, A. Rokem, K.N. Kay and B.A. Wandell (2013). Statistical evaluation of white matter connections. Annual meeting of the International Society for Magnetic Resonance in Medicine, Salt Lake City, UT and annual meeting of the Organization for Human Brain Mapping, Seattle, WA.

\item H. Takemura, F. Pestilli, A. Rokem, J. Winawer, J.D. Yeatman and B.A. Wandell (2013). The visual dorsal and ventral streams communicate through the vertical occipital fasciculus. Annual meeting of the Organization for Human Brain Mapping, Seattle, WA.

\item A. Mezer, J.D. Yeatman, A. Rokem and B.A. Wandell (2013). Language white matter tract laterality: from tractography to biophysical meaning. Annual meeting of the Organization for Human Brain Mapping, Seattle, WA.

\item J.D. Yeatman, A. Mezer, A. Rokem, F. Pestilli, H. Feldman and B.A. Wandell (2013) Automated Fiber-tract Quantification of White Matter Tissue Biology. Annual meeting of the Organization for Human Brain Mapping, Seattle, WA.

\item A. Rokem, and Landau A.N. (2013). Voluntary attention does not alleviate orientation specific surround suppression. Vision Sciences Society annual meeting. Naples, FL.

\item A. Rokem, and M.A. Silver Cholinergic enhancement increases information content of stimulus representations in human visual cortex (2012). Presentation at the symposium: Neuromodulatory Mechanisms, New Orleans, LA.

\item E. McDevitt, B. Bays, A. Rokem, M. A. Silver and S.C. Mednick (2012). Men need a nap to show perceptual learning of motion direction discrimination, but women do not. Vision Sciences Society Meeting, Naples, FL.

\item A. Rokem, Michael A. Silver (2012). Cholinergic enhancement of perceptual learning in the human visual system. Oral presentation at a symposium on neuromodulation of visual perception. Vision Sciences Society Meeting, Naples, FL.

\item A. Rokem, Michael A. Silver, Elizabeth A. McDevitt and Sara C. Mednick (2011), The effects of naps on the magnitude and specificity of perceptual learning of motion direction discrimination. Vision Sciences Society Meeting, Naples, FL.

\item A. Rokem, R.E. Ooms, J.H. Yoon, M.J. Minzenberg, C.S. Carter and M.A. Silver (2010), Broader tuning for stimulus orientation in patients with schizophrenia. Annual Meeting of the Society for Neuroscience, San Diego, CA.

\item A. Rokem, F. Perez, M. Trumpis, P. Ivanov, K. Koepsell, T. Blanche, D. Fegen, M. D'Esposito (2010). Nitime: an open-source library for time-series analysis of neuroscience data. Annual meeting of the Organization for Human Brain Mapping, Barcelona, Spain.

\item W. Prinzmetal, A. Rokem, A.N. Landau, D. Wallace, M.A. Silver, M. D’Esposito (2010). The effects of the D2 dopamine receptor agonist bromocriptine on voluntary and involuntary spatial attention in humans. Vision Sciences Society Meeting, Naples FL.

\item A. Rokem and M.A. Silver (2010) Cholinergic enhancement augments perceptual learning in the human visual system: a pharmacological fMRI study. Vision Sciences Society Meeting, Naples, FL.

\item A. Rokem and F. Perez (2009). Time-series analysis in Nipy. The 8th Python in Science Conference (SciPy 2009), Pasadena, CA

\item A. Rokem, D. Garg, A. Landau, W. Prinzmetal and M.A. Silver (2009). Effects of cholinergic enhancement on voluntary and involuntary attention. Vision Science Society Meeting, Naples, FL, Annual Meeting of the Society for Neuroscience, Chicago, IL and CSAIL conference, Hood River, OR.

\item D.W. Bressler, A. Rokem, M.A. Silver (2009). Visual spatial attention improves fMRI response reliability by decreasing the amplitude of endogenous slow oscillations in visual cortex. Annual Meeting of the Society for Neuroscience, Chicago, IL.

\item A. Rokem and M.A. Silver (2008) Cholinergic enhancement augments perceptual learning in the human visual system. Annual Meeting of the Society for Neuroscience, Washington, DC.

\item M.A. Silver, J. Yoon, A. Rokem, M.J. Minzenberg and C.S. Carter (2008) Reduced orientation-specific surround suppression in schizophrenia. Annual Meeting of the Society for Neuroscience, Washington, DC.

\item J.-H. Schleimer, A. Rokem, M.B. Stemmler (2008) Optimal phase dynamics of oscillatory neurons, the spike-triggered stimulus covariance, and maximal information transfer. Annual Meeting of the Society for Neuroscience, Washington, DC.

\item A. Rokem, S. Sanghvi and M. Silver (2007). Motion adaptation bandwidth anisotropies in the human visual system. The Optical Society of America Fall Vision Meeting, Berkeley, CA, September 2007 and Dynamical Neuroscience XV – 3rd Annual Computational Cognitive Neuroscience Conference, San Diego, CA, November 2007 (selected to appear in a special issue of Brain Research devoted to Computational Cognitive Neuroscience).

\item I. Samengo, H.G. Eyherabide, A. Rokem and A.V.M. Herz (2006). Information transmission in burst spiking. 2nd Bernstein Symposium for Computational Neuroscience. Berlin, Germany, 2006

\item M. Nahum, A. Rokem, I. Nelken and M. Ahissar (2004). Speech intelligibility \& binaural interactions: effects of stimulus familiarity, stimulus similarity \& set size. The annual meeting of the Israeli Society for Neuroscience, the 30th Goettingen Neurobiology conference, and Cosyne 2005.

\item A. Rokem and M. Ahissar (2004). Interactions between sensory and cognitive abilities in early-blind individuals. The annual meeting of the Israeli Society for Neuroscience and at the 30th Goettingen Neurobiology conference.

\item S. Watzl, A. Rokem, T. Gollisch and A.V. Herz (2003). Coding capacitites of auditory receptor cells under different stimulus conditions. The 29th Goettingen Neurobiology conference.
\end{enumerate}


\section*{Teaching}
\years{10/2017} Guest instructor -- Data Science and Society, UW Department of Sociology (Instructor: Afra Mashhadi)

\years{3/2016} Instructor -- Software Carpentry Instructor Training, eScience Institute.

\years{3/2015} Lead Instructor -- Software Carpentry, eScience Institute

\years{1/2015} Lead Instructor -- Software Carpentry, eScience Institute

\years{11/2015} Instructor -- Data Carpentry (neuroimaging), Indiana University Psychology Department.

\years{11/2015} Guest Instructor -- eScience Python Seminar (Instructor: Jake Vanderplas).

\years{10/2015} Instructor -- Software Carpentry, eScience Institute.

\years{5/2015} Instructor -- Organization for Human Brain Mapping, Brainhacking 101.

\years{11/2014} Instructor -- Software Carpentry, Stanford University.

\years{8/2014} Instructor -- Software Carpentry, Washington University, St. Louis.

\years{4/2014} Guest Instructor: MA capstone class, Department of Statistics, University of California, Berkeley (Instructor: Victoria Stodden).

\years{3/2014} Instructor -- Python for Neuroscience Workshop, University of Nottingham, UK.

\years{10/2013} Guest Instructor -- MRI methods, Department of Psychology, Stanford University (Instructor: Brian Wandell).

\years{9/2013} Instructor -- Software Carpentry, University of Southern California.

\years{6/2013} Instructor -- Software Carpentry, Christchurch University, NZ.

\years{4/2013} Instructor -- Software Carpentry, Lawrence Berkeley National Lab.

\years{3/2013} Lead Instructor -- Software Carpentry, Neuroscience Graduate Program, Stanford University.

\years{10/2012} Instructor -- Reproducible Research in Neuroimaging Workshop. Stanford Center for Cognitive and Neurobiological Imaging.

\years{8/2007} Instructor -- Matlab and the Psychophysics Toolbox, Department of Psychology, UC Berkeley.

\years{Spring 2008} Teaching Assistant -- Brain Mind and Behavior, Department of Biology, UC Berkeley (Instructor: David Presti).

\years{Fall 2006} Teaching Assistant -- Mammalian Neuroanatomy, Department of Biololgy, UC Berkeley (Instructor: Jeff Winer).

\years{Fall 2003} Teaching Assistant -- Perception, Department of Psychology, Hebrew University of Jerusalem (Instructor: Merav Ahissar).

\section*{Mentorship}
\years{2017 -- } Postdoc mentor (with Jason Yeatman): Anisha Keshavan, UW Institute for Learning and Brain Science, eScience Institute and The UW Institute for Neurongineering

\years{2017} PhD committe member: Kivan Polimis, UW Department of Sociology.

\years{2017 -- } PhD mentor (with Ione Fine): Ezgi Y\"{u}cel, UW Department of Psychology, eScience Institute and The UW Institute for Neurongineering

\years{2016 -- 2017} Postdoc mentor (with Magda Balazinska): Dongfang Zhao, UW Department of Computer Science and Engineering, and UW eScience Institute.

\years{2016 -- } Postdoc mentor (with Ione Fine): Michael Beyeler, UW Department of Psychology, eScience Institute and The UW Institute for Neurongineering

\years{Summer 2016} Shahnawaz Ahmed -- Google Summer of Code Student (Dipy).

\years{Summer 2015} Rafael Neto-Henriques -- Google Summer of Code Student (Dipy).

\years{2013} Kimberly Chan -- Stanford University research internship.

\years{2007 - 2011} UC Berkeley Undergraduate Research Apprenticeship Program: Vanessa Hoffman, Dave Garg, Hong Chao-Chun, Jon Kelvey, Greg Lam, Matthew Koh, Kimberly Chan.

\years{Summer 2007} Shradha Sanghvi -- UC Berkeley School of Optometry NEI Summer Research Training Program.

\section*{Service}

Editor, Special Topic: Explicating the interplay between anatomical and functional connectivity in the human brain, and Review Editor \emph{Frontiers in Human Neuroscience}

\vspace{4pt}

Editorial board member for \emph{Journal of Open Source Software} and \emph{Journal of Open Research Software}

Program committee member for \emph{Pattern Recognition in Neuroimaging} (2015, 2016), \emph{Scientific Computing in Python} (2016, 2017).

\vspace{4pt}

Chair (together with Olivia Guest) of mini-symposium in neuroscience,
\emph{Scientific Computing in Python} (2017).

\vspace{4pt}

Reviewer for \emph{Annals of Applied Statistics}, \emph{PLoS One}, \emph{Human Brain Mapping}, \emph{Journal of Cognitive Neuroscience}, \emph{Frontiers in Human Neuroscience}, \emph{Frontiers in Abnormal Psychology}, \emph{Journal of Open Research Software}, \emph{Neuroimage}, \emph{Journal of Vision}, \emph{F1000 Research}, \emph{Journal of Neuroimaging}, \emph{Current Opinion in Neuroscience}, \emph{Psychophysiology}.

\vspace{4pt}

\years{2017 --} Member of the International Neuroinformatics Coordinating Facility Training and Education Committee.

\years{2016 -- } Software Carpentry Instructor Trainer: training and certifying instructors for Software Carpentry:

\years{2016 --} Course Director: Summer Institute in Neuroimaging and Data Science.

\years{2014} Co-organizer (with Franco Pestilli): The White Matter Matters: Diffusion MRI in Vision Science. Symposium at the Vision Sciences Society annual meeting.

\years{2012 --}  Software Carpentry Instructor.

\years{2007 - 2008} Coordinator, Working Group on Philosophy of Mind, Townsend Center for the Humanities, University of California, Berkeley.

% Footer
\bigskip
\begin{center}
  \begin{footnotesize}
    Last updated: \today
  \end{footnotesize}
\end{center}

\nobibliography{arokemCV}
\end{document}
